%- HandOut Flag ----------------------
\makeatletter
\@ifundefined{ifHandout}{%
  \expandafter\newif\csname ifHandout\endcsname
}{}
\makeatother

%- D0cum3nt ----------------------------------------------------------------------------------------------
%\documentclass[beamer,10pt]{standalone}   
\documentclass[beamer,10pt,handout]{standalone}  \Handouttrue  

\ifHandout
	\setbeameroption{show notes} %print notes   
\fi

	
%- Packages ----------------------------------------------------------------------------------------------
\usepackage{custom-style}
\usepackage{math}



%--Beamer Style-----------------------------------------------------------------------------------------------
\usetheme{toninus}
\usepackage{animate}
\usetikzlibrary{positioning, arrows}
\usetikzlibrary{shapes}


%- Bibliography (Biber) ----------------------------------------------------------------------------------
\usepackage[backend=biber,style=alphabetic,maxnames=2]{biblatex}
\bibliography{bibfile.bib}

%===========================================================%
\begin{document}
%===========================================================%
\checkpoint

%-----------------------------------------------------------%
\begin{frame}[fragile]{Abstract Hamiltonian Pairs}
  \begin{quote}
    \emph{Rogers construction naturally generalizes to a purely algebraic setting.}
  \end{quote}
  \vfill\pause

  Data:
  \begin{itemize}
    \item $G = (G,\wedge,\lbrace \cdot, \cdot \rbrace)$ a Gerstenhaber algebra;
    \item $V = (V,\d,\iota,\Lie)$ a BV-module over $G$;
    \item $\omega \in V^{k{+}1}$ a fixed cocycle, i.e. $\d \omega = 0$, for $k\geq 1$.
  \end{itemize}
  \vfill\pause

  \begin{defblock}[Hamiltonian Pairs]
    Elements of the graded vector space:
	  $$
		\Ham^0(V, \omega) := \left\{ (\alpha, X) \in V^{k-1} \oplus G^1 \mid \iota_X \omega = -\d\alpha \right\},
    $$
  \end{defblock}
  \vfill\pause

  \begin{remblock}[$\Ham^0(V, \omega)$ can also be viewed as the fibered product \( V^{k-1} \times_{V^k} G^1 \)]
     over the following pullback in the category of vector spaces:
	\begin{displaymath}
		\begin{tikzcd}
			\Ham^0(V, \omega) \arrow[r] \arrow[d] \ar[dr,"\lrcorner",very near start,phantom]& G^1 \ar[d,"\iota_{\blank}\omega"] \\
			V^{k-1} \arrow[r, "-\d"] & V^{k}
		\end{tikzcd}
	\end{displaymath}
  \end{remblock}
\end{frame}
\note[itemize]{
  \item The construction of the Roger's $L_\infty$-algebra  can be carried out for any BV-module equipped with a fixed $k+1$-cocycle $\omega$, i.e., for which $\d \omega = 0 $ with respect to the differential $\d$, since being a BV-module provides the structure of a complete Cartan calculus, which is all we need. 
}
%-----------------------------------------------------------%


%-----------------------------------------------------------%
\begin{frame}{Abstract {$L_\infty$}-algebra of Observables}


\begin{defblock}[{$L_\infty$}-algebra of Observables]
  The graded vector space $\Ham(V, \omega)$  given by
  $$
  \Ham(V, \omega)^i := \begin{cases}
    V^{k-1 + i} & i \leq -1, \\
    \Ham^0(V, \omega) & i = 0,\\
    0 & i \geq 1~;
  \end{cases} 
  $$
  with degree $(2-j)$ multibrackets $l_j\colon \Ham^{\otimes j}(V, \omega) \to \Ham(V, \omega)[2-j]$, given by:

	\begin{itemize}
		\item $
			      l_1(\alpha) = \begin{cases}
				      \d\alpha      & \text{if } i < -1, \\
				      (\d\alpha, 0) & \text{if } i = -1, \\
				      0             & \text{if } i = 0;
			      \end{cases} $\\
		\item $
        l_2\big( (\alpha_1, X_1), (\alpha_2, X_2) \big) = \big( \iota_{X_1} \iota_{X_2} \omega, \{X_1, X_2\} \big);
		      $\\
		\item $
          l_{j\geq 3}\big( (\alpha_1, X_1), \dots, (\alpha_j, X_j) \big) = -\iota_{X_1} \dots \iota_{X_j} \omega;
		      $\\
		\item All multibrackets of arity greater than or equal to $2$ vanish when evaluated on at least one degree nonzero element.
	\end{itemize}
\end{defblock}
%


\end{frame}
\note[itemize]{
    \item 	The verification that the above definition yields an honest $L_\infty$-algebra can be directly adapted from~\cite[Thm. 5.2]{rogers2012linfty} noticing that their proof is purely algebraic in nature and relies only on the Cartan calculus axioms.
}
%-----------------------------------------------------------%

%-----------------------------------------------------------%
\begin{frame}[fragile]{Abstract {$L_\infty$}-algebra of Observables (examples)}

  \begin{exblock}[Observables $L_\infty$-algebra associated with a Lie--Rinehart algebra]
    Let $(A, \mathfrak{L})$ be a Lie--Rinehart algebra, $\omega \in \CE(\mathfrak{L})^{k+1}$ a cocycle.
    \\
    We get an $L_\infty$-structure on the complex
	\begin{displaymath}
		\begin{tikzcd}[]
			A \arrow[r] &
			\mathfrak{L}^* \arrow[r] &
			\cdots \arrow[r] &
			(\Lambda^{k-2} \mathfrak{L})^* \arrow[r] &
			(\Lambda^{k-1} \mathfrak{L})^* \times_{(\Lambda^k \mathfrak{L})^*} \mathfrak{L}
		\end{tikzcd}~.
	\end{displaymath}
\end{exblock}
  \vfill\pause

\begin{exblock} 
	When 
   $$ A=C^\infty(M)~, \qquad  \mathfrak{L}=\mathfrak{X}(M)$$
   we retrieve the Roger's construction (given in the context of closed differential forms).
\end{exblock}

\end{frame}
\note[itemize]{
    \item b
}
%-----------------------------------------------------------%




%-----------------------------------------------------------%
\ifstandalone
% https://en.wikibooks.org/wiki/LaTeX/Bibliographies_with_biblatex_and_biber
\begin{frame}[t,allowframebreaks]{Bibliography}
	%\nocite{*}
	\printbibliography
\end{frame}
\fi
%-----------------------------------------------------------%



%-----------------------------------------------------------+
\end{document}
%-----------------------------------------------------------+

%===========================================================%





We now turn to the second main part: **reduction** of the $L_\infty$-algebra of observables in the presence of constraints or symmetries. In symplectic geometry, reduction is accomplished by the Marsden-Weinstein quotient: one restricts to the zero level set of a momentum map and then quotients by the symmetry group. Algebraically, this corresponds to first imposing ideal relations (for the constraints) and then modding out by gauge symmetries. The **constraint triple** formalism packages these two steps (subobject and quotient) in a single algebraic structure .

- **Constraint Triples:** A **constraint triple** (or **coisotropic triple**) in general consists of:
    1. An algebra (or other structure) $A$ representing the full system’s observables.
    2. A subalgebra $C \subset A$ representing the *constraints* (for example, functions vanishing on a certain submanifold, or first-class constraint algebra in Dirac’s theory).
    3. A quotient (factor) algebra $\overline{A} = A/ I$ for some ideal $I$ related to $C$, representing the *reduced* algebra of observables after constraints are imposed and quotiented.
    
    These are equipped with structure maps (inclusions and projections) satisfying compatibility conditions (e.g. $I$ is the ideal of relations defining $C$) . Intuitively, one has $C$ as the "constraint algebra" (functions that vanish on the constraint surface) and $\overline{A}$ as the algebra of functions on the quotient of that constraint surface by symmetries. Dippel, Esposito & Waldmann formalized this in the context of Poisson algebras and deformation quantization , showing that many reduction procedures can be understood as functors on such triples.
    
    We adopt this idea in our setting: we will consider **constraint Gerstenhaber algebras** and **constraint BV-modules**, which are triples $(G, G_C, \overline{G})$ and $(M, M_C, \overline{M})$ capturing the full, constraint, and reduced parts of our algebraic Cartan calculus . All the structure (brackets, differentials, etc.) is assumed to exist on each part and to be compatible. For instance, a *constraint Gerstenhaber algebra* means $G$ is a Gerstenhaber algebra, $G_C \subset G$ a subalgebra (constraints), and $\overline{G} = G/G_C$ (some quotient) such that the wedge and bracket operations map triples to triples (roughly $[![G, G]!]$ sends $G_C$ to $G_C$ and induces well-defined operations on the quotient) . Similarly, a *constraint BV-module* $(M, M_C, \overline{M})$ has $M_C$ a submodule (the  "constraints on forms ") and $\overline{M}=M/M_C$ the quotient module, with differential $d$ and contractions $i_X$ etc. respecting the triples (so that, e.g., $d(M_C)\subseteq M_C$ and induces a differential on $\overline{M}$) . This is a bit of structure to keep track of, but conceptually it means we have *Cartan calculus not just on one algebra, but simultaneously on a sub- and a quotient algebra*.
    
    - **Origin in Geometry:** If one has a manifold $M$ with a coisotropic submanifold $C\subset M$ (the constraint surface) and a group $G$ acting with gauge orbits, one can form a *constraint manifold* in the sense of Dippel-Kern 2025 . The algebra of smooth functions on a constraint manifold is exactly a constraint triple $(A, C, \overline{A})$ in which $C$ are the functions vanishing on the constraint surface and $\overline{A}$ functions on the quotient (when nice). Likewise, vector fields tangent to $C$ etc. form a constraint Lie-Rinehart algebra, and so on. Indeed, it has been shown that a **constraint Lie-Rinehart algebra** naturally induces a constraint Gerstenhaber algebra and constraint BV-module structures . So this framework is not lacking examples - it generalizes classical constrained Hamiltonian systems.
- **Constraint $L_\infty$ of Observables:** Once we have constraint versions of $G$ and $M$ **and** a *constraint cocycle* $c = (c, c_C, \overline{c})$ (meaning $c$ lies in $M$, $c_C$ in $M_C$ and $\overline{c}$ is its class in $\overline{M}$, with each closed under their respective differentials), we can perform the same construction as before *in the category of constraint complexes*. That is, we define the **constraint Hamiltonian pairs** as triples $((X, X_C, \overline{X}), (m, m_C, \overline{m}))$ such that $i_X(c) = d m$, $i_{X_C}(c_C) = d m_C$ and these are compatible/projection of the first (so $\overline{X}$ and $\overline{m}$ are the images of $X,m$, etc.) . All such triples form a constraint graded space of observables. **Theorem C** asserts that this constraint observables space carries a **constraint $L_\infty$-algebra** structure . By  "constraint $L_\infty$-algebra, " we mean an $L_\infty$ whose underlying structure is also a triple $(L, L_C, \overline{L})$: there are brackets $\ell_r$ on each of $L, L_C, \overline{L}$, and the inclusion/projection maps intertwine these (so that $L_C$ is an $L_\infty$-subalgebra and $\overline{L}$ the quotient $L_\infty$) . The brackets are defined using the same formulas as in the unconstrained case, applied levelwise to $(G, M, c)$, $(G_C, M_C, c_C)$ and inducing $(\overline{G}, \overline{M}, \overline{c})$ . The $L_\infty$ identities hold simultaneously on each level by the same reasoning as before, thanks to the compatibility of the constraint data.
- **Reduction Functor:** Once we have a constraint $L_\infty$-algebra of observables, *reducing it* means passing to the **quotient part** of the constraint triple. In categorical terms, there is a forgetful functor sending a constraint object $(X, X_C, \overline{X})$ to the quotient $\overline{X}$. When we apply this to the constraint $L_\infty$ of observables, we obtain an $L_\infty$-algebra on $\overline{L}$, which we interpret as the **reduced $L_\infty$-algebra of observables** of the constrained system. Intuitively, this $\overline{L}$ encodes observables that are invariant under the symmetry and defined on the constraint surface (since we mod out those that vanish or differ by constraint terms) . By construction, all the $L_\infty$ brackets descend to $\overline{L}$, so the homotopy Lie structure is well-defined on the quotient. This algebra $\overline{L}$ is the algebraic analog of the Poisson algebra of the reduced phase space in classical reduction, but here it can handle the homotopy brackets arising from a multisymplectic form and even if no nice manifold quotient exists (e.g. in singular cases).
- **Recovery of Known Results:** Our general reduction scheme, when applied to the **geometric case** of a multisymplectic manifold with a group action, reproduces the results of our earlier work (Blacker-Miti-Ryvkin, SIGMA 2024) . In that work, we constructed a reduced $L_\infty$ of observables by more ad-hoc means; now we see it fits into the constraint triple framework. Specifically, given a Lie group $G$ acting on an $n$-plectic manifold $(M,\omega)$ with a covariant momentum map $\mu$ , one can form the constraint triple:
    - $A = $ observables $L_\infty$ on $(M,\omega)$,
    - $A_C = $ those observables that vanish on the constraint (e.g. forms pulled back by $\mu$ setting to zero, etc.),
    - $\overline{A} = $ observables on the reduced space (when $0$ level set mod $G$ is nice, these correspond to forms on the quotient).
    
    Our formalism shows that $\overline{A}$ inherits an $L_\infty$-algebra structure automatically . No assumptions of regularity or freeness are needed in the algebraic approach - even if the quotient is singular, $\overline{A}$ is well-defined as a homotopy algebra. This extends the multisymplectic reduction of Blacker 2021 (which assumed regularity) to a far more general context . Moreover, within our algebraic setting we identify an object called the **residue defect**, which measures the slight difference between naive and actual reduction in the homotopy context. In the earlier work, this residue term appeared somewhat mysteriously to satisfy the homotopy Jacobi identities; here we can interpret it cleanly as arising from the constraint differential and the failure of a certain exactness on the constraint submodule. Our framework **clarifies the role of the  "residue defect "**: it is an intrinsic part of the constraint $L_\infty$ structure that vanishes under appropriate conditions (like when the momentum map level is regular) .
    
- **Summary of Main Results:** We can summarize the main technical results (theorems) as follows:
    - **Theorem A:** *Given any Gerstenhaber algebra $G$ with a BV-module $M$ and a closed element $c\in M$, the graded space of Hamiltonian pairs $\mathcal{O}(G,M;c)$ admits an $L_\infty$-algebra structure (observables algebra), defined by the multilinear maps derived from Cartan calculus.* 
    - **Theorem B:** *If $(A,E)$ is a **constraint** Lie-Rinehart algebra (constraint version of derivations over a constraint algebra $A$), then one naturally obtains a constraint Gerstenhaber algebra and a constraint BV-module $(G, M)$ out of it. (In short, classical constraint manifolds yield constraint Cartan calculus algebraically.)
    - **Theorem C:** *Given a constraint Gerstenhaber algebra and BV-module with a constraint cocycle $c$, one can construct a **constraint** $L_\infty$-algebra of observables. Moreover, its **reduced part** $\overline{\mathcal{O}}$ (the quotient level) is an $L_\infty$-algebra that describes the observables of the reduced system.* 
    
    As an application, Theorem C applied to multisymplectic manifolds with symmetries recovers the reduced $L_\infty$ in  and explains the extra terms (residue defect) in a categorical way .
    

## **Conclusion and Outlook**

We have developed a **purely algebraic toolkit** for constructing and reducing $L_\infty$-algebras of observables, extending concepts from symplectic geometry to multisymplectic and even noncommutative settings . This framework shows the power of combining homotopy algebra (for observables) with constraint category theory (for reduction). It provides a unified perspective where classical geometric reduction is just one instance of an algebraic functor acting on $L_\infty$-algebras .

Some directions for further research include:

- **Homotopy Momentum Maps:** Relating our constraint $L_\infty$-reduction to the notion of *homotopy momentum maps* (which appear in alternative higher reduction approaches, e.g. the work of Callies-Frégier-Rogers-Zambon on $L_\infty$-moment maps . It would be interesting to connect the **Leibniz-algebra-valued momentum maps** used here  with homotopy moment maps in an $L_\infty$-context.
- **Operadic Formulation:** The constraint triple constructions hint at an operadic or higher-categorical organization. One could aim to describe the entire process (Cartan calculus + constraints + $L_\infty$) in terms of operads or properads encoding these algebraic relations . This might streamline proofs and allow further generalizations (e.g. to quantum or $A_\infty$ settings).
- **Applications to Field Theory:** Ultimately, multisymplectic forms arise in classical field theory (e.g. the 2-form in mechanics vs. a 3-form in 2-dimensional field theory). Our algebraic observables $L_\infty$ could serve as a foundation for understanding *higher Poisson brackets* on the phase space of fields and their reduction under symmetries. Recent works on polysymplectic and polycontact reduction provide complementary geometric approaches; linking those with our algebraic method is a promising avenue.

**References:** *(Selected key references in context)* Rogers (2012) for the original multisymplectic $L_\infty$-algebra construction; Dippel-Esposito-Waldmann (2019) for constraint (coisotropic) triples in Poisson algebra setting; Blacker-Miti-Ryvkin (2024) for algebraic multisymplectic reduction; and our work Miti-Ryvkin (2025) (Differential Geom. Appl.) which this talk is based on, for full details and proofs. The methods presented exemplify how **pure algebraic techniques can not only recover but also enlighten classical geometric results** in higher symplectic geometry, broadening the scope to singular and noncommutative realms.