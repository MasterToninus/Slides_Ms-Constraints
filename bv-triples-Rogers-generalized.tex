%- HandOut Flag ----------------------
\makeatletter
\@ifundefined{ifHandout}{%
  \expandafter\newif\csname ifHandout\endcsname
}{}
\makeatother

%- D0cum3nt ----------------------------------------------------------------------------------------------
\documentclass[beamer,10pt]{standalone}   
%\documentclass[beamer,10pt,handout]{standalone}  \Handouttrue  

\ifHandout
	\setbeameroption{show notes} %print notes   
\fi

	
%- Packages ----------------------------------------------------------------------------------------------
\usepackage{custom-style}
\usepackage{math}



%--Beamer Style-----------------------------------------------------------------------------------------------
\usetheme{toninus}
\usepackage{animate}
\usetikzlibrary{positioning, arrows}
\usetikzlibrary{shapes}


%- Bibliography (Biber) ----------------------------------------------------------------------------------
\usepackage[backend=biber,style=alphabetic,maxnames=2]{biblatex}
\bibliography{bibfile.bib}

%===========================================================%
\begin{document}
%===========================================================%
\checkpoint

%-----------------------------------------------------------%
\begin{frame}[fragile]{Abstract Hamiltonian Pairs}
  \begin{quote}
    \emph{Rogers construction naturally generalizes to a purely algebraic setting.}
  \end{quote}
  \vfill\pause

  Data:
  \begin{itemize}
    \item $G = (G,\wedge,\lbrace \cdot, \cdot \rbrace)$ a Gerstenhaber algebra;
    \item $V = (V,\d,\iota,\Lie)$ a BV-module over $G$;
    \item $\omega \in V^{k{+}1}$ a fixed cocycle, i.e. $\d \omega = 0$, for $k\geq 1$.
  \end{itemize}
  \vfill\pause

  \begin{defblock}[Hamiltonian Pairs]
    Elements of the graded vector space:
	  $$
		\Ham^0(V, \omega) := \left\{ (\alpha, X) \in V^{k-1} \oplus G^1 \mid \iota_X \omega = -\d\alpha \right\},
    $$
  \end{defblock}
  \vfill\pause

  \begin{remblock}[$\Ham^0(V, \omega)$ can also be viewed as the fibered product \( V^{k-1} \times_{V^k} G^1 \)]
     over the following pullback in the category of vector spaces:
	\begin{displaymath}
		\begin{tikzcd}
			\Ham^0(V, \omega) \arrow[r] \arrow[d] \ar[dr,"\lrcorner",very near start,phantom]& G^1 \ar[d,"\iota_{\blank}\omega"] \\
			V^{k-1} \arrow[r, "-\d"] & V^{k}
		\end{tikzcd}
	\end{displaymath}
  \end{remblock}
\end{frame}
\note[itemize]{
  \item The construction of the Roger's $L_\infty$-algebra  can be carried out for any BV-module equipped with a fixed $k+1$-cocycle $\omega$, i.e., for which $\d \omega = 0 $ with respect to the differential $\d$, since being a BV-module provides the structure of a complete Cartan calculus, which is all we need. 
}
%-----------------------------------------------------------%


%-----------------------------------------------------------%
\begin{frame}{Abstract {$L_\infty$}-algebra of Observables}


\begin{defblock}[{$L_\infty$}-algebra of Observables]
  The graded vector space $\Ham(V, \omega)$  given by
  $$
  \Ham(V, \omega)^i := \begin{cases}
    V^{k-1 + i} & i \leq -1, \\
    \Ham^0(V, \omega) & i = 0,\\
    0 & i \geq 1~;
  \end{cases} 
  $$
  with degree $(2-j)$ multibrackets $l_j\colon \Ham^{\otimes j}(V, \omega) \to \Ham(V, \omega)[2-j]$, given by:

	\begin{itemize}[label=$\cdot$]
		\item $
			      l_1(\alpha) = \begin{cases}
				      \d\alpha      & \text{if } i < -1, \\
				      (\d\alpha, 0) & \text{if } i = -1, \\
				      0             & \text{if } i = 0;
			      \end{cases} $\\
		\item $
        l_2\big( (\alpha_1, X_1), (\alpha_2, X_2) \big) = \big( \iota_{X_1} \iota_{X_2} \omega, \{X_1, X_2\} \big);
		      $\\
		\item $
          l_{j\geq 3}\big( (\alpha_1, X_1), \dots, (\alpha_j, X_j) \big) = -\iota_{X_1} \dots \iota_{X_j} \omega;
		      $\\
		\item All multibrackets of arity greater than or equal to $2$ vanish when evaluated on at least one degree nonzero element.
	\end{itemize}
\end{defblock}
%


\end{frame}
\note[itemize]{
    \item 	The verification that the above definition yields an honest $L_\infty$-algebra can be directly adapted from~\cite[Thm. 5.2]{rogers2012linfty} noticing that their proof is purely algebraic in nature and relies only on the Cartan calculus axioms.
}
%-----------------------------------------------------------%

%-----------------------------------------------------------%
\begin{frame}[fragile]{Abstract {$L_\infty$}-algebra of Observables (examples)}

  \begin{exblock}[Observables $L_\infty$-algebra associated with a Lie--Rinehart algebra]
    Let $(A, \mathfrak{L})$ be a Lie--Rinehart algebra, $\omega \in \CE(\mathfrak{L})^{k+1}$ a cocycle.
    \\
    We get an $L_\infty$-structure on the complex
	\begin{displaymath}
		\begin{tikzcd}[]
			A \arrow[r] &
			\mathfrak{L}^* \arrow[r] &
			\cdots \arrow[r] &
			(\Lambda^{k-2} \mathfrak{L})^* \arrow[r] &
			(\Lambda^{k-1} \mathfrak{L})^* \times_{(\Lambda^k \mathfrak{L})^*} \mathfrak{L}
		\end{tikzcd}~.
	\end{displaymath}
\end{exblock}
  \vfill\pause

\begin{exblock} 
	When 
   $$ A=C^\infty(M)~, \qquad  \mathfrak{L}=\mathfrak{X}(M)$$
   we retrieve the Roger's construction (given in the context of closed differential forms).
\end{exblock}

\end{frame}
\note[itemize]{
    \item b
}
%-----------------------------------------------------------%




%-----------------------------------------------------------%
\ifstandalone
% https://en.wikibooks.org/wiki/LaTeX/Bibliographies_with_biblatex_and_biber
\begin{frame}[t,allowframebreaks]{Bibliography}
	%\nocite{*}
	\printbibliography
\end{frame}
\fi
%-----------------------------------------------------------%



%-----------------------------------------------------------+
\end{document}
%-----------------------------------------------------------+

%===========================================================%


