%- HandOut Flag ----------------------
\makeatletter
\@ifundefined{ifHandout}{%
  \expandafter\newif\csname ifHandout\endcsname
}{}
\makeatother

%- D0cum3nt ----------------------------------------------------------------------------------------------
\documentclass[beamer,10pt]{standalone}   
%\documentclass[beamer,10pt,handout]{standalone}  \Handouttrue  

\ifHandout
	\setbeameroption{show notes} %print notes   
\fi

	
%- Packages ----------------------------------------------------------------------------------------------
\usepackage{custom-style}
\usepackage{math}



%--Beamer Style-----------------------------------------------------------------------------------------------
\usetheme{toninus}
\usepackage{animate}
\usetikzlibrary{positioning, arrows}
\usetikzlibrary{shapes}


%- Bibliography (Biber) ----------------------------------------------------------------------------------
\usepackage[backend=biber,style=alphabetic,maxnames=2]{biblatex}
\bibliography{bibfile.bib}

%===========================================================%
\begin{document}
%===========================================================%
\checkpoint


%-----------------------------------------------------------+
\subsection{Lie--Rinehart Algebras}
%-----------------------------------------------------------+

%-----------------------------------------------------------+
\begin{frame}{Lie-Rinehart Algebras}
  
  \begin{quote}
    \centering
    \emph{... generalize the structure of vector fields acting as derivations ...}
  \end{quote}
 
  \begin{defblock}[Lie-Rinehart Algebra]
    A \underline{triple} $(A,\mathfrak{L},\rho)$ where
    \begin{itemize}[label=$\triangleright$]
      \item $A$ is a commutative, associative algebra,
      \item $\mathfrak{L}$ is an $A$-module and a Lie algebra,
      \item there is an anchor map $\rho: \mathfrak{L} \to \mathrm{Der}(A)$ satisfying the Leibniz rule:
      
      \vspace{-.5em}
        $$
          [X, aY] = a[X,Y] + (\rho(X)a)Y, \quad \forall X,Y \in \mathfrak{L}, a \in A.
        $$
    \end{itemize}

  \end{defblock}  
  \pause\vfill

  \begin{exblock}[Derivations]
    $$
      A=\text{any commutative algebra}~, \qquad \mathfrak{L}=\mathrm{Der}(A)~,\qquad \rho=\mathrm{id}
    $$ 
  \end{exblock}
  \pause\vfill

  \begin{exblock}[Vector fields]
    $$
      A=C^\infty(M)~,  \qquad \mathfrak{L}=\Gamma(TM)~,\qquad \rho=\mathrm{id}
    $$ 
    \begin{quotation}
      \centering
      Lie-Rinehart algebras are algebraic analogues of Lie algebroids
      \\
      (and every Lie algebroid gives a Lie-Rinehart algebra)
    \end{quotation} 
  \end{exblock}
  \vfill

\end{frame}
\note[itemize]{
    \item Let ${A}$ be a commutative  algebra over a field $\mathbb{k}$.
  Denote by $\X(A)=\Der_{\mathbb{k}}(A)$ its Lie algebra of derivations, which naturally forms an $A$-submodule of $\End(A)$ with the action $A\action \End(A)$ given by postmultiplying with the element of $A$.
  \item In the context of differential geometry, a primary example is when $A = C^\infty(M)$, the algebra of smooth functions on a smooth manifold $M$. In this case, $\mathfrak{X}(A)$ is isomorphic to the standard $C^\infty(M)$-module of smooth vector fields on $M$, often denoted by $\mathfrak{X}(M)$.
  \item To generalize the above geometric structures, we introduce the algebraic counterparts of  "multivector fields " and  "differential forms with Cartan calculus. " 
  \item The anchor map $\rho: E \to \mathop{\mathrm{Der}}(A)$ abstractly encodes the notion of derivations of $A$ with an $A$-linear Lie bracket. 
  \item   In fact, Lie-Rinehart algebras are algebraic analogues of Lie algebroids (and every Lie algebroid gives a Lie-Rinehart algebra).
  \item singular foliations in the sense of Androulidakis and Skandalis can be viewed as Lie--Rinehart algebras
  \item a Lie--Rinehart algebra over $C^{\infty}(M)$ that is projective as a module is equivalent to a Lie algebroid,  
}
%-----------------------------------------------------------+


%-----------------------------------------------------------+
\subsection{Gerstenhaber Algebras and Modules}
%-----------------------------------------------------------+


%-----------------------------------------------------------+
\begin{frame}{Gerstenhaber Algebras}

  \begin{quote}
    \centering
    \emph{... generalize the structure of the exterior algebra of multivector fields ...}
  \end{quote}
  %
  \begin{defblock}[Gerstenhaber Algebra]
    A \underline{triple} $(G, \wedge, \lbrace\cdot,\cdot\rbrace)$ where
    \begin{itemize}[label=$\triangleright$]
		\item $G$ is a $\mathbb{Z}$-graded vector space;
		\item $\wedge : G \otimes G \to G$ is a graded commutative, associative product;
		\item $\{\blank,\blank\} :G[1]\otimes G[1]\to G[1]$ is a graded Lie bracket,
  \end{itemize}
Compatibility: The bracket is a derivation of $\wedge$:
	\begin{align*}\label{eq:Gerstenhaber-derivationProperty}
		\{a, b \wedge c\} = \{a, b\} \wedge c + (-1)^{(|a|-1)|b|} b \wedge \{a, c\}
		&& \forall~ a, b, c \in G.
	\end{align*}

  \end{defblock}  
    \vfill\pause
 

    \begin{exblock}[Exterior algebra of a Lie-Rinehart algebra]
      $G = \Lambda^\bullet \mathfrak{L}$ for a Lie-Rinehart algebra $(A,\mathfrak{L},\rho)$ with $\lbrace\cdot,\cdot\rbrace$ the Schouten-Nijenhuis bracket extending the Lie bracket on $E$.
    \end{exblock}
 
 
\end{frame}
\note[itemize]{
  \item  The bracket $\{\blank,\blank\}$ could also be reinterpreted as a degree $-1$ bracket on the desuspended $\Z$-graded vector space $G$, coincides with the definition of $1$-Poisson algebra of \cite{Cattaneo2006b}.
  \item The Schouten--Nijenhuis bracket is the unique graded Lie bracket satisfying the following properties:
\begin{itemize}
	\item It restricts to the given Lie bracket on $\Lambda_A^1 \mathfrak{L} = \mathfrak{L}$;
	\item It vanishes on $A = \Lambda_A^0 \mathfrak{L}$;
	\item It satisfies the graded Leibniz rule with respect to the wedge product.
\end{itemize}
%
Explicitly, for homogeneous elements $x_1, \dots, x_m, y_1, \dots, y_n \in \mathfrak{L}$ and $f \in A$, it is given by
\begin{align*}
	[x_1 \wedge \cdots \wedge x_m,\; y_1 \wedge \cdots \wedge y_n] 
	&= \sum_{i,j} (-1)^{i+j} [x_i, y_j] \wedge x_1 \wedge \cdots \widehat{x_i} \cdots \wedge x_m \wedge y_1 \wedge \cdots \widehat{y_j} \cdots \wedge y_n~.
\end{align*}
}
%-----------------------------------------------------------+

 %-----------------------------------------------------------+
\begin{frame}{Gerstenhaber Modules}
  %
  \begin{quote}
    \centering
    \emph{... generalize the structure of (Grassmann) algebra of differential forms...}
  \end{quote}
  %
  \begin{defblock}[Gerstenhaber Module ~~{\small( over a Gers. Alg. $G$ )}]
    A \underline{graded vector space} $V$ equipped with a \emph{Gerstenhaber algebra structure} \\ on $G\oplus V^{\mathrm{rev}}$ such that:
    \begin{itemize}[label=$\triangleright$]
      \item $G \hookrightarrow G\oplus V^{\mathrm{rev}}$ is a  Gerstenhaber subalgebra,
      \item Both brackets and associative multiplication vanish when restricted to $V^{\rev}$
    \end{itemize}
  \end{defblock}
  \vfill\pause

  \begin{exblock}[Chevalley--Eilenberg Algebra]
    The Chevalley--Eilenberg algebra for the Lie-Rinehart algebra $(A,\mathfrak{L},\rho)$:
    $$
    \CE(\mathfrak{L}) := \left((\Lambda_A^\bullet \mathfrak{L})^*\right)^{\rev}
    $$ 
    \smallskip
    
    carries a canonical structure of Gerstenhaber module over $\Lambda_A^\bullet \mathfrak{L}$. 
  \end{exblock}

\end{frame}
\note[itemize]{
  \item  meta-concept: given a type of algebraic structure $\Gamma$ (e.g., associative, Poisson, Gerstenhaber, or Lie algebra) and a $\Gamma$-algebra $A$, the data of a module structure over $A$ can often be encoded as a $\mathbb{k}$-vector space $M$ together with a $\Gamma$-structure on $A \oplus M$, which:
	\begin{itemize}
		\item the inclusion $A \hookrightarrow A \oplus M$ is a $\Gamma$-algebra morphism,
		\item squares to zero, i.e., the structure is trivial when restricting everything to $M$.
	\end{itemize}
  \item The two grading reversals from in the example cancel out, i.e. we have a Gerstenhaber algebra structure on $\Lambda\mathfrak{L}\oplus(\Lambda\mathfrak{L})^*$. However, if one omits both reversals, the formulas and gradings one obtains become inconsistent with the literature.
  \item The algebra $\CE^\bullet(\mathfrak{L})$ is canonically endowed with the wedge product
  \item $*$ stands for the $A$-linear dual. $\Hom_A(\Lambda_A^k \mathfrak{L}, A)$.
  \item $\rev$ indicates the reversal of the grading, i.e., $(V^{\rev})^k = V^{-k}$.  
}
%-----------------------------------------------------------+



%-----------------------------------------------------------+
\subsection{Cartan Calculus}
%-----------------------------------------------------------+

%-----------------------------------------------------------+
\begin{frame}{BV-Modules}
  %
  \begin{quote}
    \centering
    \emph{... generalize the structure of (the de Rham) algebra ...}
  \end{quote}
  %
  \begin{defblock}[Batalin-Vilkovisky Module]
    Is a \underline{graded vector space} $V$ together with
    \begin{itemize}[label=$\triangleright$]
      \item a differential $\d: V^i \to V^{i+1}$ (i.e. $d^2=0$),
      \item a Gerstenhaber module structure over a Gerstenhaber algebra $(G, \wedge, \lbrace\cdot,\cdot\rbrace)$,
    \end{itemize}
    Compatibility wrt the Lie algebra action $\Lie:G\action V$: 
    $$
      \Lie_x = [\iota_x, \d]~, \qquad \forall x \in G~.
    $$
  \end{defblock}
  \vfill\pause

    \begin{exblock}[Chevalley--Eilenberg Algebra]
    The Chevalley--Eilenberg algebra  $\CE(\mathfrak{L})$
    carries a canonical differential
\begin{align*}
	\dCE(\alpha)&(x_1 \wedge \dotsb \wedge x_{k+1})
  \qquad\qquad\qquad {\tiny(
  \forall \alpha \in \CE^k(\mathfrak{L})~, x_1, \dotsc, x_{k+1} \in \mathfrak{L})}
  \\
  =&~\phantom{+} \sum_{i=1}^{k+1} (-1)^{i+1} \rho(x_i)\big(\alpha(x_1 \wedge \dotsb \widehat{x_i} \dotsb \wedge x_{k+1})\big) \\
	&~+ \sum_{1 \leq i < j \leq k+1} (-1)^{i+j} \alpha\big(\lbrace x_i, x_j\rbrace \wedge x_1 \wedge \dotsb \widehat{x_i} \dotsb \widehat{x_j} \dotsb \wedge x_{k+1}\big)~,\\
  &~.
\end{align*}
  \end{exblock}

\end{frame}
\note[itemize]{
  \item usually $\d$ is called \emph{BV differential}.
  \item The algebra $\CE^\bullet(\mathfrak{L})$ is canonically endowed with the wedge product and a differential $\dCE \colon \CE^k(\mathfrak{L}) \to \CE^{k+1}(\mathfrak{L})$. The notation $\widehat{X_i}$ indicates omission of the $i$-th term from the wedge product.  
  
}
%-----------------------------------------------------------+


%-----------------------------------------------------------+
\begin{frame}{Abstract Cartan Calculus}
  %
  \begin{quote}
    \centering
    \emph{BV-module $(V,\d)$ over $(G,\wedge,\lbrace\cdot,\cdot\rbrace)$ encode a full Cartan calculus}
  \end{quote}
  %
  \vfill\pause
  \begin{denotblock}
    \begin{itemize}
      \item $\d$ the degree $1$ differential on $V$;
	    \item $\iota$ the associative algebra action\footnote{$\iota_{x\wedge y}=\iota_x\iota_y~, \qquad \forall x,y\in G$.} of $(G,\wedge)$ on $V$;
  	  \item $\Lie$ the Lie algebra action\footnote{$\Lie_{x\wedge y}=\iota_x\Lie_y + (-1)^{|y|}\Lie_x \iota_y~, \qquad \forall  x,y\in G$.} of $(G,\{\blank,\blank\})$ on $V$ of degree $-1$.
    \end{itemize}   
  \end{denotblock}
  \vfill\pause

  \begin{lemblock}[Cartan identities]
 The operators $\d$, $\iota_x$, and $\Lie_x$ satisfy six graded commutation relations:
	%
  \begin{align*}
	  \d^2                            &= 0,            &&&
	\iota_X \iota_Y + \iota_Y \iota_X  &= 0 ~,         \\
\iota_X \d   +	\d \iota_X      &= \Lie_X ~,    &&&
	\d \Lie_X - \Lie_X \d          &= 0 ~,    \\
	\Lie_X \iota_Y - \iota_Y \Lie_X    &= \iota_{[X,Y]}~,  &&&
	\Lie_X \Lie_Y - \Lie_Y \Lie_X      &= \Lie_{[X,Y]}~. 
\end{align*}
	%
  \end{lemblock}
  \vfill\pause

  \begin{exblock}[Classical Cartan Calculus]
    \vspace{-0.5em}
    $$ A=C^\infty(M)~,  \quad \mathfrak{L}=\X(M)~,\quad
     G=\Lambda^\bullet \X(M)~, \quad V=\Omega^\bullet(M)~. $$
  \end{exblock}
  %
\end{frame}
\note[itemize]{
  \item 	A pair $(G, V)$ consisting of a Gerstenhaber algebra and a BV module is often called a \emph{Tsygan–Tamarkin–Nest noncommutative differential calculus}~\cite[Def.~4.4]{Tsygan2004}.
  \item Denote Gerstenhaber algebra structure on $G \oplus V^{\rev}$ by $(\star,\llbracket\blank,\blank\rrbracket)$. For a homogeneous element $x \in G$, denote by $\iota_x$ and $\Lie_x$ the corresponding operators on $V$.
  \item These actions are induced by restricting the Gerstenhaber structure $(\star,\llbracket\blank,\blank\rrbracket)$ on $G \oplus V^{\rev}$ to $G \otimes V^{\rev} \subset (G \oplus V^{\rev})^{\otimes 2}$.
 \item The associative action $\iota$ is defined by
	\[
	\morphism{\iota}{G \otimes V^{\rev}}{V^{\rev}}{(x,\alpha)}{\iota_x \alpha := x \star \alpha},
	\]
	and satisfies
	\[
	\iota_{x \wedge y} \alpha = \iota_x \iota_y \alpha \quad \text{and} \quad \iota_x \alpha = (-1)^{|x||\alpha|} \alpha \star x.
	\]

	\item The Lie action $\Lie$ is defined by
	\[
	\morphism{\Lie}{G \otimes V^{\rev}}{V^{\rev}[-1]}{(x,\alpha)}{\Lie_x \alpha := \llbracket x, \alpha \rrbracket},
	\]
	and satisfies
	\[
	\Lie_{\{x,y\}} \alpha = \Lie_x \Lie_y \alpha + (-1)^{(|x|-1)(|y|-1)} \Lie_y \Lie_x \alpha,
	\]
	as well as the graded antisymmetry identity
	\[
	\Lie_x \alpha = -(-1)^{|x||\alpha|} \llbracket \alpha, x \rrbracket.
	\]
 \item The compatibility between $\star$ and $\llbracket\blank,\blank\rrbracket$ in $G \oplus V^{\rev}$ implies the following \emph{mixed Leibniz rules} (see \cite[Def.~2.1]{Kowalzig2015}):
  \begin{align}
	  \iota_{\{x,y\}} \alpha &= \Lie_x \iota_y \alpha - (-1)^{(|x|-1)|y|} \iota_y \Lie_x \alpha, \\
	  \Lie_{x \wedge y} \alpha &= \iota_x \Lie_y \alpha + (-1)^{|x||y|} \iota_y \Lie_x \alpha. \label{eq:mixedleib}
  \end{align}
  \item The use of the reverse grading on $V$ in the definition ensures that for any homogeneous $x \in G^k$, the operators $\iota_x$ and $\Lie_x$ have degrees $-k$ and $1-k$, respectively, when viewed as endomorphisms of $V$.
}
%-----------------------------------------------------------+


%-----------------------------------------------------------%
\ifstandalone
% https://en.wikibooks.org/wiki/LaTeX/Bibliographies_with_biblatex_and_biber
\begin{frame}[t,allowframebreaks]{Bibliography}
	%\nocite{*}
	\printbibliography
\end{frame}
\fi
%-----------------------------------------------------------%


%-----------------------------------------------------------+
\end{document}
%-----------------------------------------------------------+




