




%- HandOut Flag -----------------------------------------------------------------------------------------
\newif\ifHandout
%	\Handouttrue  %uncomment for the printable version

%- D0cum3nt ----------------------------------------------------------------------------------------------
\documentclass[beamer,handout,10pt]{standalone}   
\ifHandout
	\setbeameroption{show notes} %print notes   
\fi

	
%- Packages ----------------------------------------------------------------------------------------------
\usepackage{custom-style}
\usepackage{math}

%--Beamer Style-----------------------------------------------------------------------------------------------
\usetheme{toninus}
\usepackage{math}





%--------------------------------------------------------------------------------------------------
%- D0cum3nt ----------------------------------------------------------------------------------------------------------------------------------
\begin{document}
%------------------------------------------------------------------------------------------------


%------------------------------------------------------------------------------------------------
\begin{frame}{From symplectic to {multi}symplectic} 
	%
	\begin{center}
		$-$ \emph{multisymplectic means \textbf{going higher} in the degree of $\omega$} $-$
	\end{center}
	\pause
	\begin{defblock}[$n$-plectic manifold ~\emph{(Cantrijn, Ibort, De Le\'on)} \cite{Cantrun2017}]
		\includestandalone[width=0.95\textwidth]{Pictures/Figure_multisym}	
	\end{defblock}
	%
	\pause
		\begin{table}
			\begin{tabular}{c c c}
				symplectic forms \small($n=1$) & $\leftrightsquigarrow$ & volume forms \small($n= \text{dim}(M)-1$)
			\end{tabular}
		\end{table}

	\vfill
	\pause
	\begin{block}{Historical motivation}
		Mechanics: geometrical foundations of \textit{(first-order)} field theories.
	\end{block}
	%
	\begin{table}
		\ifHandout
		%
		\else
		\only<4>{
		\begin{tabular}{|p{0.2\textwidth}|p{0.3\textwidth}|p{0.35\textwidth}|} 
            \hline
            \parbox[][20pt][c]{0.2\textwidth}{mechanics} & \multicolumn{2}{c|}{geometry} \\
            \hline
            \parbox[][20pt][c]{0.2\textwidth}{phase space} & symplectic manifold &  \\[.25em]
            \parbox[][20pt][c]{0.2\textwidth}{classical \\ observables} & Poisson algebra &  \\[.25em]
            \parbox[][20pt][c]{0.2\textwidth}{symmetries} &  group actions admitting comoment map &  
            \\
            \hline
  \multicolumn{1}{c}{}
            &  \multicolumn{1}{@{}c@{}}{$\underbrace{\hspace*{.3\textwidth}}_{\text{point-like particles systems}}$} 
            &  \multicolumn{1}{@{}c@{}}{}              \\
		\end{tabular}
		}
		\fi
		\onslide<5->{
		\begin{tabular}{|p{0.2\textwidth}|p{0.3\textwidth}|p{0.35\textwidth}|} 
            \hline
            \parbox[][20pt][c]{0.2\textwidth}{mechanics} & \multicolumn{2}{c|}{geometry} \\
            \hline
            \parbox[][20pt][c]{0.2\textwidth}{phase space} & symplectic manifold & multisymplectic manifold \\[.25em]
            \parbox[][20pt][c]{0.2\textwidth}{classical \\ observables} & Poisson algebra & $L_\infty$-algebra \\[.25em]
            \parbox[][20pt][c]{0.2\textwidth}{symmetries} &  group actions admitting comoment map & group actions admitting (homotopy) comomentum map
            \\
            \hline
  			\multicolumn{1}{c}{}
            &  \multicolumn{1}{@{}c@{}}{$\underbrace{\hspace*{.3\textwidth}}_{\text{point-like particles systems}}$} 
            &
            \multicolumn{1}{@{}c@{}}{$\underbrace{\hspace*{.3\textwidth}}_{\text{field-theoretic systems}}$} 
               \\
		\end{tabular}
		}
	\end{table}	




\end{frame}
\note[itemize]{
	\item Historically, the interest in multisymplectic manifolds, has been motivated by the need for understanding the geometrical foundations of first-order classical field theories.
	The key point is that, just as one can associate a symplectic manifold to an ordinary classical mechanical system (e.g. a single
point-like particle constrained to some manifold), it is possible to associate a multisymplectic
manifold to any classical field system (e.g. a continuous medium like a filament or a fluid). See frame Extra-\ref{Frame:Ms-Field-Mechanics} 
	
	\item General ideas basic parallelisms with caveats
	\item caveat: points in multiphase spaces are not states
	\item the table hides the duality between geometric and algebraic approaches to the problem.
	\item	Mechanics: geometrical foundations of \textit{(first-order)} field theories.
		\begin{itemize}
		 \item[-] Kijowski, W. Tulczyjew \cite{Kijowski1979}; %(1979)
		 \item[-] Cariñena, Crampin, Ibort \cite{Carinena1991b};% (1991)
		 \item[-] Gotay, Isenberg, Marsden, Montgomery \cite{Gimmsy1};%(1998)
		 \\ $\cdots$
		\end{itemize}
}
%------------------------------------------------------------------------------------------------

%------------------------------------------------------------------------------------------------
\begin{frame}{Observables in $n$-plectic geometry}
	%
	\begin{defblock}[Hamiltonian $(n-1)$-forms]
		\begin{displaymath}
			\Omega^{n-1}_{ham}(M,\omega) 	:=
			\biggr\{ \sigma \in  \Omega^{n-1}(M) \; \biggr\vert \; 
				\exists \vHam_\sigma \in \mathfrak{X}(M) ~:~ 
				\tikz[baseline,remember picture]{\node[rounded corners,
                        fill=orange!5,draw=orange!30,anchor=base]            
            			(target) {$d \sigma = -\iota_{\vHam_\sigma} \omega$ };
            	}				
				~\biggr\} 
			\end{displaymath}
	\end{defblock}

	\pause
		\tikz[overlay,remember picture]
		{
			\node[rounded corners,
                 fill=orange!5,draw=orange!30,anchor=base]
            	 (base) at ($(current page.north east)-(2,1)$) [rotate=-0,text width=3.5cm,align=center] {\footnotesize{\textcolor{red}{Hamilton-DeDonder-Weyl \\equation}}};
		}	
	\begin{tikzpicture}[overlay,remember picture]
    	\path[->] (base.south) edge[bend right,red](target.north);
    \end{tikzpicture}
	%
	\vfill
	\begin{columns}[T]
		\pause
		\setlength{\belowdisplayskip}{5pt}
		\begin{column}{.50\linewidth}
			%
			\vspace{-1em}
			\begin{thmblock}[Observables $L_\infty$-algebra]
				$\Omega^{n-1}_{ham}(M,\omega)$ endowed with
				\vspace{-.5em}
				\begin{displaymath}
					\lbrace \sigma_1, \sigma_2 \rbrace =			
					~ - \iota_{\vHam_1}\iota_{\vHam_2} \omega 
				\end{displaymath}			
				can be "completed" to a \\ $L_\infty-algebra$.
			\end{thmblock}
			%	
		\end{column}	
		%
		\pause
		\begin{column}{.50\linewidth}
		%
			\begin{itemize}
				\item[\cmark] Skew-symmetric;
				\item[\xmark] multiplication of observables;
				\item[\xmark] Jacobi equation;
				%\\ \hspace*{4.25em} full-fledged Jacobi equation;
				\item[\smark] Jacobi equation \emph{up to homotopies}.
			\end{itemize}			
			%
				\[
					\small
					\{\alpha,\{\beta,\gamma\}\} + \text{\it cyc. perm.}
					= {\color{orange}\d \iota_{X_\alpha} \iota_{X_\beta} \iota_{X_\gamma}\omega}
				\]
		\end{column}	
	\end{columns}
	%
	\vfill
	\pause
	Interesting alternatives:
	\begin{itemize}
		\item  descend to $\Omega^{k-1}_{ham}(M) / {\color{orange}\d\hspace{1pt}\Omega^{k-2}(M)}$, hence getting a Lie algebra;
		\item Incorporate the {\color{orange}discrepancy} as part of the data of the space of observables $ 
				L_\infty(M,\omega) = \Omega_{ham}^{k-1}(M)\oplus{\color{orange}\Omega^{\leq k-2}(M)}$
			%a \textbf{homotopy Lie algebra} or \textbf{$L_\infty$-algebra}.
	\end{itemize}
	
	
\end{frame}
\note[itemize]{
	\item
}
%------------------------------------------------------------------------------------------------




%--------------------------------------------------------------------------------------------------
\subsection{$L_\infty$-algebra of Observables}
\begin{frame}[fragile,t]{$L_\infty$-algebra of Observables (higher observables) }
	Let be $(M,\omega)$ a $n$-plectic manifold.
	\begin{defblock}[$L_\infty$-algebra of observables ~\emph{(Rogers)} ~\cite{Rogers2010}]
		
		\hspace{.25em} $L\infty(M,\omega)$ is given by:
		
		\begin{itemize}
			\item[•] a cochain-complex $(L,\{\cdot\}_1)$ 
		\end{itemize}
		\begin{center}
		\ifHandout
			\includestandalone{Pictures/Figure_Observables}	
		\else
			\includestandalone{Pictures/Frame_Observables}
		\fi				
		\end{center}
		\onslide<2->{
			\begin{itemize}
				\item[•] with $n$ (skew-symmetric) multibrackets $(2 \leq k \leq n+1)$
			\end{itemize}
			\begin{center}
				\includestandalone{Pictures/Equation_Multibracket}	
			\end{center}
		}
		%
	\end{defblock}
  \end{frame}
 \note[itemize]{
	\item if symplectic manifolds are the symmetric take on mechanics, Poisson algebras are the algebraic counterpart.
 	\item A Lie algebra is associated to an ordinary symplectic manifold (its Poisson algebra).
	%(Underlying this is a Lie algebra, whose Lie bracket is the Poisson bracket.)
	Similarly, one associates an Lie-$n$ algebra to any $n$-plectic manifold.
 	% https://ncatlab.org/nlab/show/n-plectic+geometry 	 
 	 %https://ncatlab.org/nlab/show/Poisson+bracket+Lie+n-algebra
	 \item Basically, the higher observables algebra is a chunk of the de Rham complex of $M$ with inverted grading( convention employed here) and an extra structure called "multibrackets".
 	\item ( In the 1-plectic case it reduces to the corresponding Poisson algebra of classical observables)
 	\item Rogers associated to any n-plectic mfd a $L\-\infty$ algebra, Zambon generalized it to the pre-n-plectic case.
 	\item Recognize in the definition of $\{\cdot,\ldots,\cdot\}_k$ the contraction with hamiltonian fields $v_\sigma$ w.r.t. $\sigma$.
  	\item Note $	\iota_{v_{\sigma_1}}\cdots\iota_{v_{\sigma_k}} = (-)^{(k-1)+(k-2)+\dots+1}\iota_{v_{\sigma_k}}\cdots\iota_{v_{\sigma_1}} = (-)^{\frac{k(k-1)}{2}}\iota_{v_{\sigma_k}}\cdots\iota_{v_{\sigma_1}}$ 
 	The definition usually find in literature of Rogers multibrackets involves the coefficient $ (-)^{\frac{k(k-1)}{2}} = -\varsigma(k-1) = (-)^{k+1} \varsigma(k)$.
  \item higher observables is Special instance of a more general object  called $L\-\infty$ Algebra...
 }
%------------------------------------------------------------------------------------------------








%------------------------------------------------------------------------------------------------
\ifstandalone
\begin{frame}[t,shrink]{Extended Bibliography}
	\cite{Blacker20}
	\bibliographystyle{alpha}
	\bibliography{bibfile}
\end{frame}
\fi
%------------------------------------------------------------------------------------------------

%------------------------------------------------------------------------------------------------
\end{document}