%- HandOut Flag -----------------------------------------------------------------------------------------
\newif\ifHandout
%	\Handouttrue  %uncomment for the printable version

%- D0cum3nt ----------------------------------------------------------------------------------------------
\documentclass[beamer,handout,10pt]{standalone}   
\ifHandout
	\setbeameroption{show notes} %print notes   
\fi

	
%- Packages ----------------------------------------------------------------------------------------------
\usepackage{custom-style}

%--Beamer Style-----------------------------------------------------------------------------------------------
\usetheme{toninus}


%\usepackage{graphicx}
\usepackage{enumitem}
\usepackage{amsmath}
\usepackage{amssymb}



\usepackage{tikz-cd}
\usepackage{tcolorbox}
\usepackage[all]{xy}
\usepackage{quiver} %https://q.uiver.app/ % Not in Miktex!
\usetikzlibrary{nfold}
\usetikzlibrary{decorations.pathmorphing} 

\newcommand{\Ham}{\mathrm{Ham}}
\newcommand{\Der}{\mathrm{Der}}



%--------------------------------------------------------------------------------------------------
%- D0cum3nt ----------------------------------------------------------------------------------------------------------------------------------
\begin{document}
%------------------------------------------------------------------------------------------------



\section{LEOS}
% TEMP! All credit to Leonid Ryvkin www.ryvkin.eu


\section{Motivation: Symplectic reductions}
\subsection{Symplectic manifolds}
\begin{frame}

\begin{block}{Def. Symplectic manifold $(M,\omega)$}
	$M$ a smooth manifold, $\omega\in \Omega^{2}(M)$ closed and non-degenerate, i.e.:
\begin{itemize} 	
	\item $d\omega=0$
	\item $\omega^\flat:TM\to T^*M, v\mapsto \iota_v\omega$ is injective
\end{itemize}	
\end{block}


Examples:\pause
\begin{itemize}
	\item orientable 2-manifolds with volume, e.g. $S^2$, $T^2$, $\mathbb R^2$\pause
	\item cotangent bunlde $T^*Q$ of any manifold $Q$, with $\omega=d\theta$. $\theta_\eta(v)=\eta(T\pi_Q(v))$.\pause
	\item coadjoint orbits
\end{itemize}

\end{frame}


\begin{frame}
	\begin{block}{Def. Poisson algebra of observables} Let $(M,\omega)$ be symplectic.
		\begin{itemize}
			\item $X_f:=-(\omega^\flat)^{-1}(df)$  \emph{Hamiltonian vector field} of $f\in C^\infty(M)$.
			\item $\{f,g\}:=\omega(X_f,X_g)=L_{X_f}g$ \emph{Poisson bracket of observables}
		\end{itemize}
	\end{block}\pause
	Indeed $(C^{\infty}(M), \cdot, \{-,-\})$ Poisson algebra:
			\begin{itemize}
		\item $\cdot$ is a commutative, associative mutliplication
		\item $\{-,-\}$ is a Lie bracket
		\item $\{f,- \}$ is a derivation
	\end{itemize}\pause
	And $f\mapsto X_f$ is a Lie algebra map. 
	
\end{frame}

\subsection{Marsden-Weinstein-Meyer reduction}
\begin{frame}[fragile]{Marsden-Weinstein-Meyer reduction}
Let $(M,\omega)$ symplectic, $G\circlearrowright M$ be a Lie group action\pause and $\mu:M\to \mathfrak g^*$ such that:\pause
\begin{itemize}
\item  $\mu$ is a momentum, i.e. the infinitesimal action $\xi\mapsto \underline\xi:\mathfrak g\to \mathfrak X(M)$ of $G$ satisfies $X_{\langle\mu,\xi\rangle}=\underline \xi$. \pause
\item $\mu$ is $G$-equivariant, i.e. $\mu(gm)=Ad^*_g\mu(m)$ \pause
\item The action of $G$ on $\mu^{-1}(0)$ is free and proper.\\ (This implies $0$ is reg. val. of $\mu$) \pause
\end{itemize}
Then $\exists!$ symplectic form $\omega_r$ on $M_r=\frac{\mu^{-1}(0)}{G}$ such that $\pi^*\omega_r=i^*\omega$.

% https://q.uiver.app/#q=WzAsMyxbMCwwLCJNIl0sWzMsMCwiXFxtdV57LTF9KDApIl0sWzYsMCwiTV9yIl0sWzEsMCwiaVxcXFwgXFxzdWJzdGFja3tpbmNsdXNpb25+b2Z+c3Vic3BhY2V9IiwyLHsic3R5bGUiOnsidGFpbCI6eyJuYW1lIjoiaG9vayIsInNpZGUiOiJib3R0b20ifX19XSxbMSwyLCJcXHBpXFxcXCBcXHN1YnN0YWNre3F1b3RpZW50fmJ5fnJlbGF0aW9ufSJdXQ==
\[\begin{tikzcd}
	M &&& {\mu^{-1}(0)} &&& {M_r}
	\arrow["\begin{array}{c} i\\ \substack{inclusion~of~subspace} \end{array}"', hook', from=1-4, to=1-1]
	\arrow["\begin{array}{c} \pi\\ \substack{quotient~by~relation} \end{array}", two heads, from=1-4, to=1-7]
\end{tikzcd}\]

\pause
Examples: $\mathbb CP^n$ from $S^1$-action on $\mathbb C^{n+1}$.


\end{frame}



\subsection{Attempting algebraic reduction}

\begin{frame}[fragile]
When the action is not 'free and proper', we can try to mimic this directly using the function algebras:
% https://q.uiver.app/#q=WzAsNixbMCwwLCJNIl0sWzEsMCwiXFxtdV57LTF9KDApPVMiXSxbMywwLCJNX3IiXSxbMCwxLCJDXlxcaW5mdHkoTSkiXSxbMSwxLCJDXntcXGluZnR5fShTKT1cXGZyYWN7Q157XFxpbmZ0eX0oTSl9e0lfe1N9fSJdLFszLDEsIkNee1xcaW5mdHl9KE1fcik9Q15cXGluZnR5KFMpXkciXSxbMSwyLCJcXHN1YnN0YWNre3F1b3RpZW50fSIsMCx7InN0eWxlIjp7ImhlYWQiOnsibmFtZSI6ImVwaSJ9fX1dLFszLDQsIlxcc3Vic3RhY2t7cXVvdGllbnR9IiwwLHsic3R5bGUiOnsiaGVhZCI6eyJuYW1lIjoiZXBpIn19fV0sWzAsMSwiXFxzdWJzdGFja3tyZXN0cmljdGlvbn0iLDAseyJzdHlsZSI6eyJib2R5Ijp7Im5hbWUiOiJzcXVpZ2dseSJ9fX1dLFs0LDUsIlxcc3Vic3RhY2t7cmVzdHJpY3Rpb259IiwwLHsic3R5bGUiOnsiYm9keSI6eyJuYW1lIjoic3F1aWdnbHkifX19XV0=
\[\begin{tikzcd}
	M & {\mu^{-1}(0)=S} && {M_r} \\
	{C^\infty(M)} & {C^{\infty}(S):=\frac{C^{\infty}(M)}{I_{S}}} && {C^{\infty}(M_r)=C^\infty(S)^G}
	\arrow["{\substack{restriction}}", squiggly, from=1-1, to=1-2]
	\arrow["{\substack{quotient}}", two heads, from=1-2, to=1-4]
	\arrow["{\substack{quotient}}", two heads, from=2-1, to=2-2]
	\arrow["{\substack{restriction}}", squiggly, from=2-2, to=2-4]
\end{tikzcd}\]

where $I_S=\{f\in C^{\infty}(M)~|~f|_S=0\}$.\pause Equivalently:

% https://q.uiver.app/#q=WzAsMyxbMCwwLCJDXlxcaW5mdHkoTSkiXSxbMiwwLCJOX1M9XFx7Zn58fiBmXFxpbiBDXlxcaW5mdHkoTSl8X1N+fkdcXG1hdGhybXstaW52YXJpYW50fVxcfSJdLFs0LDAsIlxcZnJhY3tOX1N9e0lfU30iXSxbMCwxLCJcXHN1YnN0YWNre3Jlc3RyaWN0aW9ufSIsMCx7InN0eWxlIjp7ImJvZHkiOnsibmFtZSI6InNxdWlnZ2x5In19fV0sWzEsMiwiXFxzdWJzdGFja3txdW90aWVudH0iLDAseyJzdHlsZSI6eyJoZWFkIjp7Im5hbWUiOiJlcGkifX19XV0=
\[\small\begin{tikzcd}
	{C^\infty(M)} && {N_S=\{f~|~ f\in C^\infty(M)|_S~~G\mathrm{-invariant}\}} && {\frac{N_S}{I_S}}
	\arrow["{\substack{restriction}}", squiggly, from=1-1, to=1-3]
	\arrow["{\substack{quotient}}", two heads, from=1-3, to=1-5]
\end{tikzcd}\]
\pause

\begin{alertblock}{Problem}
	$C^\infty(S)^G=\frac{N_S}{I_S}$ is not a Poisson algebra in general.
\end{alertblock}
(when constraints not first class, cf. Dirac reduction)
\end{frame}

 \begin{frame}{Other approaches to reduction}
 
\begin{table}[]
	\begin{tabular}{l|l|l||l}
		algebra $\mathcal{A}_T$ & subalgebra $\mathcal{A}_N$ & ideal $\mathcal{A}_0$ & comment \\
		\hline
		
		$C^\infty(M)$&      $N_S$     &   $I_S$   &  $\substack{S=\mu^{-1}(0),~ I_S=\{f~|~f|_S=0\}}$ \\
			&&&  $\substack{N_S= \{ f | gf-f\in I_S \forall g\in G\}}$ \\
	& & & \color{red} not Poisson \\ 
	\hline	\pause
	$C^\infty(M)$&   $F_S$         &   $I_S$   &  $\substack{F_S=\{f~|~\{f,I_S\}\subset I_S\}}$ \\
	& & & \color{red} if $I_S\subset F_S$\\
	\hline \pause
	
	$C^\infty(M)$&   $C^\infty(M)^G$         &   $(I_S)^G$   &  $\substack{C^\infty(M)^G=\{f~|~gf=f\forall g\in G\}}$ \\
	\hline \pause
		$C^\infty(M)$&   $C^\infty(M)^G$         &   $(I_\mu)^G$   & $\substack{I_\mu=\{\sum_{i=1}^k f_i\langle \mu,\xi_i\rangle|~f_i\in C^\infty(M), \xi_i\in \mathfrak g\}}$\\
		\hline \pause
	$C^\infty(M)$&      $N_\mu$     &   $I_\mu$   & $\substack{N_\mu= \{ f | gf-f\in I_\mu \forall g\in G\}}$\\
	\hline 
		
	\end{tabular}\pause
\end{table}
{\bf Remarks }names in order: 'naive' , Dirac, Sniatycki, Arms-Cushman-Gotay, Sniatycki-Weinstein.\pause \\
When $G$ connected $N_\mu=N(I_\mu):=\{f~|~\{f,I_\mu\}\subset I_\mu\}$. 
 \end{frame}
 
 
 

\subsection{Preview of our answer}
\begin{frame}
We will develop a multisymplectic reduction in this talk, symplectically it is:
\begin{block}{Construction: Multisymplectic reduction in the symplectic case}
Let $Q_S=\{f~|~ f|_S=0\rm{~and~} df|_S=0\}$ Then the $L_\infty$-reduction is:
$$
\frac{N(I_S)\cap N(I_\mu+Q)}{I_\mu+Q}
$$
\end{block}	\pause
Such a formula can be more generally applied to a Poisson algebra $P$ with associative ideals $I\subset J$ such that $\{I,I\}\subset I$ and $\{I,J\}\subset J$:
$$
\frac{N(J)\cap N(I+Q)}{I+Q}
$$
with $Q=\{f\in J|~\{f,P\}\subset J\}$
\end{frame}



\section{Multisymplectic manifolds and their observable algebras}
\subsection{Dramatis Personae}
\begin{frame}
\begin{block}{Def. Multisymplectic manifold $(M,\omega)$}
	$M$ a smooth manifold, $\omega\in \Omega^{n+1}(M)$ closed and non-deg., i.e.:
	\begin{itemize}
		\item $d\omega=0$
		\item $\omega^\flat:TM\to \Lambda^nT^*M, v\mapsto \iota_v\omega$ is injective
	\end{itemize}	\pause
\end{block}
Without the non-deg. condition we call it \emph{premultisymplectic}.	\pause
\begin{itemize} 
	\item orientable $n{+}1$-manifolds with volume, e.g. $S^{n+1}$, $T^{n+1}$, $\mathbb R^{n+1}$\pause
	\item multicotangent bunlde $\Lambda^nT^*Q$ of any manifold $Q$, with $\omega=d\theta$. $\theta_\eta(v_1,...,v_n)=\eta(T\pi_Q(v_n),...,T\pi_Q(v_n))$.\pause
	\item semisimple Lie groups, hyperKähler manifolds,...
\end{itemize}

\end{frame}


\subsection{Motivating example of $L_\infty$-algebra}

\begin{frame}{Back to the symplectic case, $M=\mathbb R^2$}
	Let $\omega=dx\wedge dy$, $\mathfrak X(\mathbb R^2,\omega)=\{X|~L_X\omega=0\}$.\pause
	
	\begin{itemize}
		\item {\bf Symplectic gradient}\\  $sgrad:C^\infty(\mathbb R^2)\to \mathfrak X(\mathbb R^2,\omega)$ (surjecive, kernel $\mathbb R$):\\ 
		$$f\mapsto sgrad(f)={-\partial^yf \choose \partial^xf}$$\pause 
		\item {\bf Poisson bracket} on $C^\infty(\mathbb R^2)$: $$\{f,g\}:=\partial^xf\partial^yg-\partial^xg\partial^yf$$ \\pause
		
		\item {\bf Preservation of the bracket} $$sgrad(\{f,g\})=[sgrad(f),sgrad(g)]$$
	\end{itemize}\pause
	{\color{teal} Conclusion: } One can use the nicer space $C^\infty(\mathbb R^2)$ instead of $\mathfrak X(\mathbb R^2,\omega)$!
\end{frame}





\begin{frame}{Let's try the same in $\mathbb R^3$}
	\begin{itemize}
		\item $\mathfrak X(\mathbb R^3)=\{(v_x,  v_y,  v_z)^T\}
		$\\ 
		\item {\bf Lie bracket :}
		$$[(v_x,  v_y,  v_z)^T,(w_x,  w_y,  w_z)^T]= (\sum_i v_i\partial^i(w_x) - w_i\partial^i(v_x) ,  \ldots ,\ldots)^T  $$
		\item {\bf volume form: }$\Omega=dx\wedge dy\wedge dz$ 
		\item {\bf volume-preserving fields: }$$\mathfrak X(\mathbb R^3,\Omega)=\{(v_x,v_y,v_z)^T~| \sum_i\partial^iv_i=0\}$$
	\end{itemize}\pause~\\
	This time the surjection is given by the {\bf Rotational}:
	
	$$rot:\mathfrak X (\mathbb R^3)\to \mathfrak X (\mathbb R^3,\Omega)$$
	$$
	rot((v_x,v_y,v_z)^T)= (\partial^yv_z-\partial^zv_y, \partial^z v_x-\partial^xv_z, \partial^x v_y-\partial^yv_x)^T
	$$
	
	
\end{frame}

\begin{frame}
	
	{\color{red} Problem: } $rot$ does not commute with the bracket: $[rot(X),rot(Y)]\neq rot([X,Y])$ in general\\
	~\\
	\pause
	{\color{teal} Solution: } New bracket $\{X,Y\}:=rot(X)\times rot(Y)$,\\ where $\times$ is the vector cross product on $\mathbb R^3$.\\~\\
	\pause
	We have $rot(\{X,Y\})=[rot(X), rot(Y)]$ \pause but no Jacobi identity. However: $$\{X,\{Y,Z\}\}-\{\{X,Y\},Z\}-\{Y,\{X,Z\}\}$$
	$$=- grad((rot(X)\times rot(Y))\cdot rot(Z))$$
	
\end{frame}


\begin{frame}
	$C^\infty(\mathbb R^3)\oplus \mathfrak X(\mathbb R^3)=:L_{-1}\oplus L_0$ form a \underline{\bf $L_\infty$ algebra} with:
	\begin{itemize} 	
		\item $l_1:L_{-1}\to L_0$, $l_1(f)=grad(f)$
		\item $l_2:L_{0}\times L_0\to L_0$, $l_2(X,Y)=\{X,Y\}$
		\item $l_3:L_{0}\times L_0\times L_0\to L_{-1}$, $l_3(X,Y,Z)=-(rot(X)\times rot(Y))\cdot rot(Z)$ 
	\end{itemize}~\\~\\ \pause
	In our case it means the following identities are satisfied:
	\begin{itemize} 	
		\item $l_2(l_1(f),Y)=0$
		\item $l_2(l_2(X,Y),Z) - cycl= l_1l_3(X,Y,Z)$
		\item $l_3(l_2(X,Y),Z,W)-cycl=0$
	\end{itemize}
\end{frame}

\subsection{The multisymplectic observable algebra}


\begin{frame}[fragile]
	\begin{block}{Def. Observable $L_\infty$ algebra [Rogers2012]}
		Let $(M,\omega)$ pre-multisymplectic. Set:
			\begin{itemize} 	
			\item $\Ham^0(M,\omega)=\{(\alpha,v)\in \Omega^{n-1}(M)\times \mathfrak X(M)~|~d\alpha=-\iota_v\omega  \}$
			\item $\Ham^{-i}(M,\omega)= \Omega^{n-1-i}(M)$, $1\leq i\leq n-1$.
			\item  $l_1=d:\Ham^{-i}(M,\omega)\to \Ham^{-i+1}(M,\omega)$,\\ (resp. $(d,0)$ for $i=1$)
			\item $l_k:\Lambda^k\Ham^0(M,\omega)\to \Ham^{2-k}(M,\omega),$
			\begin{align*}
			&l_k((\alpha_1,v_1),...,(\alpha_k,v_k))=-(-1)^{\frac{k(k+1)}{2}}\iota_{v_k}...\iota_{v_1}\omega & k\geq 3\\
			&l_2((\alpha_1,v_1),(\alpha_2,v_2))=(\iota_{v_2}\iota_{v_1}\omega, [v_1,v_2])
			\end{align*}
		\end{itemize}
	\end{block}
	\vspace{-15mm}
	\pause
	% https://q.uiver.app/#q=WzAsNixbMCwwLCJDXlxcaW5mdHkoTSkiXSxbMSwwLCJcXE9tZWdhXjEoTSkiXSxbMiwwLCIuLi4iXSxbMywwLCJcXE9tZWdhXntuLTN9KE0pIl0sWzQsMCwiXFxPbWVnYV57bi0yfShNKSJdLFs1LDAsIlxcSGFtXjAoTSkiXSxbMCwxLCJkIl0sWzEsMiwiZCJdLFsyLDMsImQiXSxbMyw0LCJkIl0sWzQsNSwiKGQsMCkiXSxbNSw1LCJsXzIiLDAseyJsYWJlbF9wb3NpdGlvbiI6NDAsIm9mZnNldCI6MSwiYW5nbGUiOjQ1LCJsZXZlbCI6Mn1dLFs1LDQsImxfMyIsMCx7ImN1cnZlIjotMywibGV2ZWwiOjN9XSxbNSwzLCJsXzQiLDIseyJjdXJ2ZSI6MywibGV2ZWwiOjR9XV0=
	\[\tiny\begin{tikzcd}
		{C^\infty(M)} & {\Omega^1(M)} & {...} & {\Omega^{n-3}(M)} & {\Omega^{n-2}(M)} & {\Ham^0(M)}
		\arrow["d", from=1-1, to=1-2]
		\arrow["d", from=1-2, to=1-3]
		\arrow["d", from=1-3, to=1-4]
		\arrow["d", from=1-4, to=1-5]
		\arrow["{(d,0)}", from=1-5, to=1-6]
		\arrow["{l_4}"', curve={height=18pt}, Rightarrow, scaling nfold=4, from=1-6, to=1-4]
		\arrow["{l_3}", curve={height=-18pt}, Rightarrow, scaling nfold=3, from=1-6, to=1-5]
		\arrow["{l_2}"{pos=0.4}, shift right, Rightarrow, from=1-6, to=1-6, loop, in=10, out=80, distance=10mm]
	\end{tikzcd}\]
\end{frame}


\begin{frame}
\begin{block}{Thm. $Ham^\bullet(M,\omega)$ is an $L_\infty$-algebra [Rogers2012]}
This means $
\substack{
l_1l_{m+1}(x_1,...,x_{m+1})=\sum_{i<j}(-1)^{i+j}l_m(l_2(x_{i},x_j),x_1,...,\widehat{x_i},...,\widehat{x_j},...,x_{m+1})}
$ for all $m$ and all possible $x_i$, where $l_0:=0$ and $l_{m+2}:=0$.
\end{block}\pause

{\bf Remark:} We could instead turn $\Ham^0(M,\omega)$ into a Left-Leibniz algebra by $[(\alpha,v)(\beta, w)]:=(L_v\beta,[v,w])$. This satisfies (left-)Jacobi, but is not skew-symmetric.\pause

\begin{block}{Remark}
 The above also works if we replace $(C^\infty(M),\mathfrak X(M))$ with any Lie-Rinehart algebra $(A, L)$ and $\omega$ with a cocycle in $(\Lambda^{n+1}_AL)^*$.
\end{block}\pause
A Lie Rinehart algebra is: 
\begin{itemize} 
	\item A commutative algebra $A$ and an A-module $L$
	\item A Lie algebra structure on $L$
	\item A Lie algebra and module homomorphism $\rho:L\to \Der(A)$
\end{itemize}
such that $[\ell, a\tilde \ell]=a[\ell,\tilde \ell]+\rho(\ell)(a) \cdot \tilde\ell$
\end{frame}







\section{Reduction using Constraint triples}


\subsection{General idea}
\begin{frame}
\begin{exampleblock}{Idea of multisymplectic observable reduction:}
	Mimic the above construction of $\Ham^\bullet$\\ with triples of the type (algebra , subalgebra, ideal).
\end{exampleblock}\pause
After [Dippel2023] we call such triples { \bf constraint triples} and their parts \emph{total space, normal space, zero space}.\\~

Let $S$ be a closed set. We describe $C^\infty(S)$ by $\frac{C^\infty(M)}{I_S}$.\\ We think of it as the constraint triple:
$$
\mathcal A=(\mathcal A_T,\mathcal A_N,\mathcal A_0)=(C^\infty(M),C^\infty(M), I_S)
$$
It is a (strong) constraint algebra,\\ i.e. $\mathcal A_0\subset \mathcal A_N\subset \mathcal A_T$ subalgebras and $\mathcal A_0\subset \mathcal A_T$ ideal.


\end{frame}


\subsection{A dictionary between algebra and geometry}
\begin{frame}{Translating geometry to algebra}

\begin{table}[]
	\begin{tabular}{l|l|l}
	 geometry & algebra & translation \\
		\hline	
	smooth manifold $M$ & comm. algebra $\mathcal A_T$  	& $\mathcal A_T=C^{\infty}(M)$\\ \hline\pause
	vector fields $\mathfrak X(M)$ & derivations $\Der(\mathcal A_T)$ &  $\Der(C^\infty(M))=\mathfrak X(M)$\\ \hline\pause
	closed subset $S\subset M$ & ideals $\mathcal A_0\subset \mathcal A_T$ & $\mathcal A_0=I_S=\{f~|~f|_S=0\}$ \\ \hline\pause
	action $G\circlearrowleft M$ & Lie algebra map & infinitesimal generators\\
	 of connected group &$\underline{\cdot}:\mathfrak g\to \Der(\mathcal A_T)$ & of the action \\ \hline \pause
	 $G$ action preserves $S$ & $\underline \xi(I_S)\subset I_S$ $\forall \xi\in\mathfrak g$& \color{red} Careful! Not equivalent! \\ \hline
	\end{tabular}\pause
\end{table}
	\begin{alertblock}{Warning}
	$\partial_x$ is algebraically preserving $S=[0,1]\subset \mathbb R=M$, even though the flow leads out of the set.
\end{alertblock}
\end{frame}




\begin{frame}{Derivations of constraint algebras}
	$\mathcal A=(\mathcal A_T,\mathcal A_N,\mathcal A_0)=(C^\infty(M),C^\infty(M), I_S)$\\~
\begin{itemize} 
	\item $\Der(\mathcal A)_T:=\Der(\mathcal A)_T$ corresponds to all vector fields\pause
	\item  $\Der(\mathcal A)_N:=\{v\in \Der(\mathcal A)~|~v(\mathcal A_N)\subset \mathcal A_N, ~v(\mathcal A_0)\subset \mathcal A_0\}$\\
	corresponds to vector fields ({\color{red}algebraically}) tangent to $S$.\\ \pause

	\item  $\Der(\mathcal A)_0:=\{v\in \Der(\mathcal A)~|~v(\mathcal A_N)\subset \mathcal A_0\}$\\
	are vector fields vanishing on $S$.\pause
		\begin{block}{Lemma: Algebraic vanishing = geometric vanishing}
		$\Der(\mathcal A)_0=\{v ~| ~v|_S=0\}=I_S\cdot \mathfrak X(M)$.
	\end{block}\pause
\end{itemize}	
\end{frame}



\begin{frame}
	$\mathcal A=(C^\infty(M),C^\infty(M), I_S)$\\
	$\Der(\mathcal A)=(\mathfrak X(M),~~\{v| v(I_S)\subset I_S\},~~ I_S\mathfrak X(M))$
		\begin{block}{Lemma: $\mathfrak L:=\Der(\mathcal A)$ is a constraint Lie-Rinehart algebra over $\mathcal A$}
			This means 
	\begin{itemize} 
		\item $\mathfrak L_T$ is a Lie Rinehart algebra over $\mathcal A_T$.
		\item $\mathfrak L$ is a strong constraint module over $\mathcal A$, i.e.\\ $\mathcal A_N\mathfrak L_N\subset \mathfrak L_N$ and $\mathcal A_0\mathfrak L_T+\mathcal A_T\mathfrak L_0\subset \mathfrak L_0$.
		\item $[\mathfrak L_N,\mathfrak L_N]\subset \mathfrak L_N$ and  $[\mathfrak L_N,\mathfrak L_0]\subset \mathfrak L_0$
	\end{itemize}
	
\end{block}\pause

\begin{exampleblock}{Corollary: $(\Lambda \mathfrak L)^*$ has a constraint Cartan calculus $d, \iota, L$}
	\begin{itemize} 
	\item $(\Lambda^k \mathfrak L)^*_T=\Omega^k(M)$
	\item $(\Lambda^k \mathfrak L)^*_N=\{\alpha~|~\alpha(v_1,...,v_k)\in \mathcal A_N~\forall v_i\in \mathfrak L_N , $ \\ ~~~~~~~~~~~~~~~~~~~~$ \alpha(v_1,...,v_{k-1},w)\in \mathcal A_0~\forall v_i\in \mathfrak L_N , w\in \mathfrak L_0 \}$
	\item $(\Lambda^k \mathfrak L)^*_0=\{\alpha~|~\alpha(v_1,...,v_k)\in \mathcal A_0~\forall v_i\in \mathfrak L_N\}$
	\end{itemize}
	
\end{exampleblock}

\end{frame}


\subsection{Introducing symmetry}
\begin{frame}
Let $F\subset \Der(\mathcal A)_N$ a Lie subalgebra  (typically $F=\underline{\mathfrak g}$). Consider:
\begin{itemize} 
	\item the constraint algebra $\mathcal B= (C^\infty(M),~~~~ \{f~|~F(f)\subset I_S\},~~~ I_S)$\pause
	\item the constraint Lie-Rinehart algebra $\mathcal L$  \[ (\mathfrak X(M),~~~~\{v ~|~ [v,F]\subset C^\infty(M)F+I_S\mathfrak X(M)\}, ~~~~ C^\infty(M)F+I_S\mathfrak X(M) )\]
	\item \pause the constaint differential forms 
	\[
	\mathcal W=(\Omega(M),~~~, \{\alpha |\iota_\xi\alpha,L_\xi\alpha\in (\Lambda \mathfrak L)^*_0 ~~\forall \xi\in F \},~~~~(\Lambda \mathfrak L)^*_0  )
	\]
	where $(\Lambda^k \mathfrak L)^*_0=\{\alpha~|~\alpha(v_1,...,v_k)\in \mathcal A_0~\forall v_i\in \mathfrak L_N\}$
\end{itemize}
\begin{block}{'Lemma': $\mathcal L$ has a Cartan calculus on $\mathcal W$}
	In particular, from the premultisymplectic form $\omega$ we obtain a constraint $L_{\infty}$-algebra.
\end{block}
\end{frame}

\subsection{The reduced $L_\infty$-algebra}
\begin{frame}[fragile]
Let $(M,\omega)$ premultisymplectic, $S\subset M$ closed and \\ $F\subset \mathfrak X(M)$ a Lie subalgebra s.th. $F(I_S)\subset I_S$\\ ~\\
The following subquotient $\frac{\Ham(M,\omega)_N}{\Ham(M,\omega)_0}$\\ of $\Ham(M,\omega)_T=\Ham(M,\omega)$ forms an $L_\infty$-algebra:

	\[\small
		\frac{
			\left\lbrace
			\begin{array}{l}
				\alpha\in \Omega^{\leq n-2}(M) ~\mathrm{, respectively}\\	
				(\alpha,X) \in \Omega^{n{-}1}(M) \oplus\mathfrak X(M)
			\end{array}
			~\left\vert~
			\begin{array}{l l}
				\iota_\xi \alpha \in (\Lambda \mathfrak L)^*_0 &\\
				L_\xi\alpha \in (\Lambda \mathfrak L)^*_0&~\forall \xi \in F\\
				\hline
				\iota_X \omega = -d \alpha &\\
				X(I_S)\subset I_S &\\
				L_\xi X \in C^{\infty}(M) F + I_S\mathfrak X(M)
				&~\forall \xi \in F\\
			\end{array}
			\right.
			\right\rbrace
		}{
			\left\lbrace
			\begin{array}{l}
				\alpha\in \Omega^{\leq n-2}(M) ~\mathrm{, respectively}\\	
				(\alpha,X) \in \Omega^{n{-}1}(M) \oplus \mathfrak X(M)
			\end{array}
			~\left\vert~
			\begin{array}{l}
				\alpha \in (\Lambda \mathfrak L)^*_0\\
				\hline
				\iota_X \omega = -d \alpha\\
				X \in C^{\infty}(M) F + I_S\mathfrak X(M)
			\end{array}
			\right.
			\right\rbrace
		}
\]
where $(\Lambda^k \mathfrak L)^*_0=\{\alpha~|~\alpha(v_1,...,v_k)\in \mathcal A_0~\forall v_i\in \mathfrak L_N\}$
\end{frame}


\subsection{Main application}
\begin{frame}[fragile]
Let $(M,\omega)$ premultisymplectic, $\underline{\cdot}:\mathfrak g\to \mathfrak X(M)$ an action and\\
 $\mu:\mathfrak g\to \Omega^{n-1}(M)$ a Leibniz momentum, i.e.
\[
X_{\mu(\xi)}=\underline \xi ~~\forall \xi ~~~~~~~\mu([\xi,\zeta])=L_{\underline \xi}\mu(\zeta)~\forall \xi,\zeta
\]
Then $F=\underline {\mathfrak g}$ and $S=\{p\in M~|~\mu(\xi)_p=0\forall \xi\in\mathfrak g\}$ satisfy the necessary condition. \pause  Moreover:
\begin{block}{Thm: The geometric case [Blacker21]}
If in the above $S$ is smooth and the foliation spannd by $\underline{\mathfrak g}|_S$ is simple, then $M_{r}=\frac{S}{\underline{\mathfrak g}|_S}$ is premultisymplectic with $\omega_r$ unique, satisfying $\pi^*\omega_r=i^*\omega$.
\[\begin{tikzcd}
	M &&& S &&& {M_r}
	\arrow["\begin{array}{c} i \end{array}"', hook', from=1-4, to=1-1]
	\arrow["\begin{array}{c} \pi \end{array}", two heads, from=1-4, to=1-7]
\end{tikzcd}\]
\end{block}
\end{frame}


\begin{frame}
\begin{block}{Lemma: Relating the reductions}	
There is a natural injection $R:\frac{\Ham(M,\omega)_N}{\Ham(M,\omega)_0}\hookrightarrow \Ham(M_r,\omega_r)$.
\end{block}
\pause 
\begin{exampleblock}{Non-surjectivity in general}
	\begin{itemize} 
	\item 	$M=\mathbb R^5$, $\omega=dt\wedge (dx_1\wedge dy_1+dx_2\wedge dy_2)$\pause 
	\item $\mathfrak g=\langle\xi\rangle, \underline{\xi}=\partial_{x_1}$, $\mu(\xi)=-y_1dt$\pause 
	\item $S=\{y_1=0\}$, $M_r\cong \mathbb R^3$ with $t,x_2,y_2$ and $\omega_r=dt\wedge dx_2\wedge dy_2$\pause 
	\item $(\alpha,v)=(x_2tdy_2,x_2\partial_t-t\partial_{x_2})$ is not in the image of $R$.\pause 
	\end{itemize}
\end{exampleblock}
{\bf Reason for non-surjectivity:} From the perspective of $M_r$ it does not make sense to require $d\alpha=-\iota_v\omega$ on all of $M$.\pause \\
{\bf Remark:} We can easily construct examples where $\omega_r$ degenerates.
\end{frame}



\subsection{Lessons learned}
\begin{frame}{Lessons learned}
	\begin{itemize} 
\item Constraint triples permit to do differential geometry/ Cartan calculus on singular subquotients,\\ ... in particular constructing observable algebras on them.\pause
\item Even if $\omega$ is non-degenerate, $\omega_r$ could degenerate, forcing us to pull along the Hamiltonian vector fields explicitly\pause
\item Coomentum maps in the sense of the Leibniz structure seem better for reducing than $L_\infty$-comomenta \pause
\item The algebraic approach can treat algebroid / singular foliation symmetries and remembers vanishing orders a the singularities. \pause
\item The reduction seems to give something novel even in the symplectic case \pause
\item A 'softer' notion of constraint $L_\infty$ algebra is necessary to obtain all observables of the reduced space.

	\end{itemize}
\end{frame}















%------------------------------------------------------------------------------------------------
\ifstandalone
\begin{frame}[t,shrink]{Extended Bibliography}
	\cite{Blacker20}
	\bibliographystyle{alpha}
	\bibliography{bibfile}
\end{frame}
\fi
%------------------------------------------------------------------------------------------------




%------------------------------------------------------------------------------------------------
\end{document}