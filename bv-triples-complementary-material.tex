%- HandOut Flag ----------------------
\makeatletter
\@ifundefined{ifHandout}{%
  \expandafter\newif\csname ifHandout\endcsname
}{}
\makeatother

%- D0cum3nt ----------------------------------------------------------------------------------------------
\documentclass[beamer,10pt]{standalone}   
%\documentclass[beamer,10pt,handout]{standalone}  \Handouttrue  

\ifHandout
	\setbeameroption{show notes} %print notes   
\fi

	
%- Packages ----------------------------------------------------------------------------------------------
\usepackage{custom-style}
\usepackage{math}



%--Beamer Style-----------------------------------------------------------------------------------------------
\usetheme{toninus}
\usepackage{animate}
\usetikzlibrary{positioning, arrows}
\usetikzlibrary{shapes}


%- Bibliography (Biber) ----------------------------------------------------------------------------------
\usepackage[backend=biber,style=alphabetic,maxnames=2]{biblatex}
\bibliography{bibfile.bib}

%===========================================================%
\begin{document}
%===========================================================%
\checkpoint

%- . - . - . - . - . - . - . - . - . - . - . - . - . - . - . - . - . - . - . - . - . - . - . - .%	
\subsection{Reduction} 
%- . - . - . - . - . - . - . - . - . - . - . - . - . - . - . - . - . - . - . - . - . - . - . - .%	

%- . - . - . - . - . - . - . - . - . - . - . - . - . - . - . - . - . - . - . - . - . - . - . - .%
\begin{frame}{Reminder: momentum maps in symplectic geometry}\label{frame:symplecticmomaps}
	Consider $\theta:G\curvearrowright M$ symplectic,~~ $\underline{\cdot}:\mathfrak{g}\to \mathfrak{X}(M)$ infinitesimal action.
	\vfill

	\begin{columns}[T]
		\setlength{\belowdisplayskip}{5pt}
		\begin{column}{.50\linewidth}
			%
			\centering \it
			\onslide<2->{
				\begin{defblock}[Equivariant moment map]
					Smooth map $$\mu:M\to \mathfrak{g}^\ast$$
					such that:
					\begin{itemize}
						\item[i.] $d\langle \mu,\xi\rangle = -\iota_{\underline{\xi}}\omega$ 
						~\qquad, $\forall \xi \in \mathfrak{g}$
						\item[ii.] $\mu \circ \theta_g = Ad_g^\ast \circ \mu$
						 \qquad\small, $\forall g \in G$
					\end{itemize}
				\end{defblock}
			}
		\end{column}	
		%
		\onslide<2->{\vrule{}}
		%
		\begin{column}{.50\linewidth}
			\onslide<3->{			
				\begin{defblock}[Comoment map]
					Linear map $$\widetilde{\mu}: \mathfrak{g}\to C^\infty(M,\omega)$$
					such that:
					\begin{itemize}
						\item[i.] $d\widetilde{\mu}(\xi) = -\iota_{\underline{\xi}}\omega$ 
						\qquad~\small, $\forall \xi \in \mathfrak{g}$
						\item[ii.] $\widetilde{\mu}([\xi,\eta]) = \lbrace\widetilde{\mu}(\xi),\widetilde{\mu}(\eta)\rbrace$ \small, $\forall \xi,\eta \in \mathfrak{g}$
					\end{itemize}
				\end{defblock}
			}
		\end{column}	
	\end{columns}	
	%
	\pause
	\vspace{1em}
	%
	\begin{columns}[]
		\setlength{\belowdisplayskip}{5pt}
		\begin{column}{.40\linewidth}
			%
			\centering \it
			\onslide<5->{
				\begin{upshotblocktitle}[Duality]
					\begin{displaymath}
						\mu(x) : \mathfrak{\xi} \mapsto \widetilde{\mu}(\xi)\big\vert_x
					\end{displaymath}
					%
					\emph{
					\small
					"duality wrt. the currying operation"					
					}
				\end{upshotblocktitle}
			}
		\end{column}	
		%
		%
		\begin{column}{.60\linewidth}
			\onslide<4->{			
				\begin{upshotblocktitle}[$\widetilde{\mu}$ as a lift]
					\begin{displaymath}
						\begin{tikzcd}[ampersand replacement = \&]
						 \& C^\infty(M,\omega) \ar[d,"\vHam"]
						 \\
						 \mathfrak{g} \ar[ur,dashed,sloped,"\widetilde{\mu}"]\ar[r,"\underline{\cdot}"'] \& \mathfrak{X}(M)
						\end{tikzcd}
					\end{displaymath}
					%
					\emph{
					\small
					"it is a lift (in the Lie category) of the infinitesimal action by the assigment of hamiltonian v.fields."					
					}
				\end{upshotblocktitle}
			}
		\end{column}	
	\end{columns}		
\end{frame}
\note[itemize]{
		\item Moment maps are instrumental to the proper formalization of reduction in symplectic geometry.
		\item consider an action preserving the symplectic form.
		\item[-] $\langle \cdot, \cdot \rangle$ is the natural pairing of $\mathfrak{g}$ and $\mathfrak{g}^\ast$.
			\item[-] $\theta_g$ is the diffeomorphism associated to $g\in G$ via $\theta:G\curvearrowright M$
			\item[-] $Ad^\ast_g$ is the coadjoint action $G\curvearrowright \mathfrak{g}^\ast$
		\item The moment map can be reexpressed as a comomentum map.
		\item Observe that ii. on the left always implies ii. on the right. The converse is trure only if $G$ is connected.
		\item $G\curvearrowright M$ is said "Hamiltonian" (see slide \ref{frame:gisthamaction} in appendix) iff $\exists$ comoment map $\widetilde{\mu}$.
		\item \emph{moment map: Describe ${\color{orange}G\curvearrowright M}$ in terms of $\color{blue}C^\infty(M)\to\X(M)$.}
		\begin{center}
		\begin{tikzpicture}[scale=1.5]
			\node[orange] (A) at (0,0) {$\g$};
			\node[gray] (A1) at (0,-.3) {$\xi$};
			\node[blue] (B) at (1,1) {$C^\infty(M)$};
			\node[gray] (B2) at (1.55,1) {$f$};
			\node[orange] (C) at (1,0) {$\X(M)$};
			\node[gray] (C1) at (1,-.3) {$\underline\xi$};
			\node[gray] (C2) at (1.55,0) {$X_f$};
	
			\path[->,blue,dashed] (A) edge node[above left] {$\langle\mu,\cdot\,\rangle$} (B);
			\path[->,orange] (A) edge (C);
			\path[->] (B) edge (C);
			\path[|->,gray] (A1) edge (C1);
			\path[|->,gray] (B2) edge (C2);
	
			\begin{scope}[xshift=3cm]
				\node at (-.32,.8) {as Lie algebras};
				\node at (-.13,.2) {i.e., {\color{blue}$\mu_\xi$} generates {\color{orange}$\underline\xi$}};
			\end{scope}
		\end{tikzpicture}
		
			\footnotesize		(\emph{moment map:} ${\color{blue}\mu:M\to\g^*}$, 
		 \emph{Hamiltonian $G$-space:} 	$(M,\omega,G,\mu)$)
		\end{center}
		\item
		 $\alpha\mapsto X_\alpha$ is a homomorphism of Leibniz algebras:
		$$\d\{\alpha,\beta\} = \d\L_{X_\alpha}\beta = \L_{X_\alpha}\iota_{X_\beta}\omega = \iota_{[X_\alpha,X_\beta]}\omega ~{\color{black!80} \implies}~
		X_{\{\alpha,\beta\}} = [X_\alpha,X_\beta]$$
}
%- . - . - . - . - . - . - . - . - . - . - . - . - . - . - . - . - . - . - . - . - . - . - . - .%



%- . - . - . - . - . - . - . - . - . - . - . - . - . - . - . - . - . - . - . - . - . - . - . - .%
\begin{frame}{Other approaches to singular reduction}
 	Many singular reduction schemes adhere to this pattern!
 	\vfill
 	\pause
\begin{table}[]
	\begin{tabular}{l|l|l||l}
		algebra $\mathcal{A}_T$ & subalgebra $\mathcal{A}_N$ & ideal $\mathcal{A}_0$ & comment \\
		\hline
		
		$C^\infty(M)$&      $N_S$     &   $I_S$   &  $\substack{S=\mu^{-1}(0),~ I_S=\{f~|~f|_S=0\}}$ \\
			&&&  $\substack{N_S= \{ f | gf-f\in I_S \forall g\in G\}}$ \\
	& & & \color{red} not Poisson \\ 
	\hline	\pause
	$C^\infty(M)$&   $F_S$         &   $I_S$   &  $\substack{F_S=\{f~|~\{f,I_S\}\subset I_S\}}$ \\
	& & & \color{red} if $I_S\subset F_S$\\
	\hline \pause
	
	$C^\infty(M)$&   $C^\infty(M)^G$         &   $(I_S)^G$   &  $\substack{C^\infty(M)^G=\{f~|~gf=f\forall g\in G\}}$ \\
	\hline \pause
		$C^\infty(M)$&   $C^\infty(M)^G$         &   $(I_\mu)^G$   & $\substack{I_\mu=\{\sum_{i=1}^k f_i\langle \mu,\xi_i\rangle|~f_i\in C^\infty(M), \xi_i\in \mathfrak g\}}$\\
		\hline \pause
	$C^\infty(M)$&      $N_\mu$     &   $I_\mu$   & $\substack{N_\mu= \{ f | gf-f\in I_\mu \forall g\in G\}}$ \footnote{When $G$ connected $N_\mu=N(I_\mu):=\{f~|~\{f,I_\mu\}\subset I_\mu\}$.}\\
	\hline 
		
	\end{tabular}\pause
\end{table}
	\vfill
	{\bf Remarks }names in order: 'naive' , Dirac, Sniatycki, Arms-Cushman-Gotay, Sniatycki-Weinstein.

\end{frame}
\note[itemize]{
	\item \emph{(Credits to \href{https://www.ryvkin.eu/}{Leonid Ryvkin} for the tex code of this slide.)}
}
%- . - . - . - . - . - . - . - . - . - . - . - . - . - . - . - . - . - . - . - . - . - . - . - .%
 



%-----------------------------------------------------------%
\ifstandalone
% https://en.wikibooks.org/wiki/LaTeX/Bibliographies_with_biblatex_and_biber
\begin{frame}[t,allowframebreaks]{Bibliography}
	%\nocite{*}
	\printbibliography
\end{frame}
\fi
%-----------------------------------------------------------%



%-----------------------------------------------------------+
\end{document}
%-----------------------------------------------------------+


